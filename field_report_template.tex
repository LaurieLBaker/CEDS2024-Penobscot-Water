% Options for packages loaded elsewhere
\PassOptionsToPackage{unicode}{hyperref}
\PassOptionsToPackage{hyphens}{url}
\PassOptionsToPackage{dvipsnames,svgnames,x11names}{xcolor}
%
\documentclass[
  letterpaper,
  DIV=11,
  numbers=noendperiod]{scrartcl}

\usepackage{amsmath,amssymb}
\usepackage{iftex}
\ifPDFTeX
  \usepackage[T1]{fontenc}
  \usepackage[utf8]{inputenc}
  \usepackage{textcomp} % provide euro and other symbols
\else % if luatex or xetex
  \usepackage{unicode-math}
  \defaultfontfeatures{Scale=MatchLowercase}
  \defaultfontfeatures[\rmfamily]{Ligatures=TeX,Scale=1}
\fi
\usepackage{lmodern}
\ifPDFTeX\else  
    % xetex/luatex font selection
\fi
% Use upquote if available, for straight quotes in verbatim environments
\IfFileExists{upquote.sty}{\usepackage{upquote}}{}
\IfFileExists{microtype.sty}{% use microtype if available
  \usepackage[]{microtype}
  \UseMicrotypeSet[protrusion]{basicmath} % disable protrusion for tt fonts
}{}
\makeatletter
\@ifundefined{KOMAClassName}{% if non-KOMA class
  \IfFileExists{parskip.sty}{%
    \usepackage{parskip}
  }{% else
    \setlength{\parindent}{0pt}
    \setlength{\parskip}{6pt plus 2pt minus 1pt}}
}{% if KOMA class
  \KOMAoptions{parskip=half}}
\makeatother
\usepackage{xcolor}
\setlength{\emergencystretch}{3em} % prevent overfull lines
\setcounter{secnumdepth}{-\maxdimen} % remove section numbering
% Make \paragraph and \subparagraph free-standing
\ifx\paragraph\undefined\else
  \let\oldparagraph\paragraph
  \renewcommand{\paragraph}[1]{\oldparagraph{#1}\mbox{}}
\fi
\ifx\subparagraph\undefined\else
  \let\oldsubparagraph\subparagraph
  \renewcommand{\subparagraph}[1]{\oldsubparagraph{#1}\mbox{}}
\fi


\providecommand{\tightlist}{%
  \setlength{\itemsep}{0pt}\setlength{\parskip}{0pt}}\usepackage{longtable,booktabs,array}
\usepackage{calc} % for calculating minipage widths
% Correct order of tables after \paragraph or \subparagraph
\usepackage{etoolbox}
\makeatletter
\patchcmd\longtable{\par}{\if@noskipsec\mbox{}\fi\par}{}{}
\makeatother
% Allow footnotes in longtable head/foot
\IfFileExists{footnotehyper.sty}{\usepackage{footnotehyper}}{\usepackage{footnote}}
\makesavenoteenv{longtable}
\usepackage{graphicx}
\makeatletter
\def\maxwidth{\ifdim\Gin@nat@width>\linewidth\linewidth\else\Gin@nat@width\fi}
\def\maxheight{\ifdim\Gin@nat@height>\textheight\textheight\else\Gin@nat@height\fi}
\makeatother
% Scale images if necessary, so that they will not overflow the page
% margins by default, and it is still possible to overwrite the defaults
% using explicit options in \includegraphics[width, height, ...]{}
\setkeys{Gin}{width=\maxwidth,height=\maxheight,keepaspectratio}
% Set default figure placement to htbp
\makeatletter
\def\fps@figure{htbp}
\makeatother

\usepackage{booktabs}
\usepackage{caption}
\usepackage{longtable}
\usepackage{colortbl}
\usepackage{array}
\KOMAoption{captions}{tableheading}
\makeatletter
\makeatother
\makeatletter
\makeatother
\makeatletter
\@ifpackageloaded{caption}{}{\usepackage{caption}}
\AtBeginDocument{%
\ifdefined\contentsname
  \renewcommand*\contentsname{Table of contents}
\else
  \newcommand\contentsname{Table of contents}
\fi
\ifdefined\listfigurename
  \renewcommand*\listfigurename{List of Figures}
\else
  \newcommand\listfigurename{List of Figures}
\fi
\ifdefined\listtablename
  \renewcommand*\listtablename{List of Tables}
\else
  \newcommand\listtablename{List of Tables}
\fi
\ifdefined\figurename
  \renewcommand*\figurename{Figure}
\else
  \newcommand\figurename{Figure}
\fi
\ifdefined\tablename
  \renewcommand*\tablename{Table}
\else
  \newcommand\tablename{Table}
\fi
}
\@ifpackageloaded{float}{}{\usepackage{float}}
\floatstyle{ruled}
\@ifundefined{c@chapter}{\newfloat{codelisting}{h}{lop}}{\newfloat{codelisting}{h}{lop}[chapter]}
\floatname{codelisting}{Listing}
\newcommand*\listoflistings{\listof{codelisting}{List of Listings}}
\makeatother
\makeatletter
\@ifpackageloaded{caption}{}{\usepackage{caption}}
\@ifpackageloaded{subcaption}{}{\usepackage{subcaption}}
\makeatother
\makeatletter
\@ifpackageloaded{tcolorbox}{}{\usepackage[skins,breakable]{tcolorbox}}
\makeatother
\makeatletter
\@ifundefined{shadecolor}{\definecolor{shadecolor}{rgb}{.97, .97, .97}}
\makeatother
\makeatletter
\makeatother
\makeatletter
\makeatother
\ifLuaTeX
  \usepackage{selnolig}  % disable illegal ligatures
\fi
\IfFileExists{bookmark.sty}{\usepackage{bookmark}}{\usepackage{hyperref}}
\IfFileExists{xurl.sty}{\usepackage{xurl}}{} % add URL line breaks if available
\urlstyle{same} % disable monospaced font for URLs
\hypersetup{
  pdftitle={Water Quality Field Report Template},
  colorlinks=true,
  linkcolor={blue},
  filecolor={Maroon},
  citecolor={Blue},
  urlcolor={Blue},
  pdfcreator={LaTeX via pandoc}}

\title{Water Quality Field Report Template}
\author{}
\date{}

\begin{document}
\maketitle
\ifdefined\Shaded\renewenvironment{Shaded}{\begin{tcolorbox}[breakable, borderline west={3pt}{0pt}{shadecolor}, enhanced, interior hidden, boxrule=0pt, frame hidden, sharp corners]}{\end{tcolorbox}}\fi

This field report template serves as a guided example of the desired
format to strive for when aiming to automate the creation of reports to
be used internally by the Water Resources Program. Each single file will
be a summary report of various sites from a specific run sampled on a
specific date.These reports are intended to be used for assessment of
accuracy and completeness in field results, and should therefore
highlight missing, repetitive or inaccurate data based on the conditions
provided in the document of control flags.

\hypertarget{field-report-header}{%
\subsection{Field Report Header}\label{field-report-header}}

Each report should be created for a single run sampled on a specific
date. These reports should then be subdivided by site to display the
data. The first header should therefore be grouped by RunCode, RunDate
and primary collector. The primary collector is the first set of
initials found in the ``Collectors'' variable.

\begin{longtable*}{lrl}
\toprule
RunCode & RunDate & PrimCollector \\ 
\midrule\addlinespace[2.5pt]
LakesEBSS & 17798 & TJM \\ 
\bottomrule
\end{longtable*}

\hypertarget{table-of-contents}{%
\subsection{Table of Contents}\label{table-of-contents}}

The following sections (site information, sample table, result table,
filter table) should be differentiated by site for the specific run,
date and collector for which this report is being created. The report
should have a table of content allowing to quickly jump to the tables
for a specific site. All sites should be displayed one after the other
to allow to scroll trough all the information for this specific run
sampled on a specific date.

\hypertarget{site-visit-information}{%
\subsection{Site Visit Information}\label{site-visit-information}}

\begin{longtable*}{cccccc}
\caption*{
{\large Site information}
} \\ 
\toprule
SiteVisitID & WaterBody & SiteCode & SiteVisitStartTime & SiteDepth & SamplingAirTemp \\ 
\midrule\addlinespace[2.5pt]
JasonS7-2018924111226 & Lakes & EBL1 & 9/24/2018 11:12:26 & 5 & 10.4 \\ 
\bottomrule
\end{longtable*}

\begin{longtable*}{cccccc}
\toprule
Weather & RivCond & WaterLevel & FoamRank & FoamSource & SiteVisitComment \\ 
\midrule\addlinespace[2.5pt]
Sunny & Choppy & Low & Few Bubbles Present & Wave Action & NA \\ 
\bottomrule
\end{longtable*}

\hypertarget{sample-table}{%
\subsection{Sample Table}\label{sample-table}}

This sample table includes an example of a way to flag data by coloring
cells in a table. Here, the green was used for a condition that was
respected. You should aim to color any violations or erroneous data
based on the quality control flag document and requests of the Penobscot
Nation Water Resources Planners.

\hypertarget{tbl-sample}{}
\begin{longtable}{cccccc}
\caption{\label{tbl-sample}Sample Table }\tabularnewline

\toprule
SampleName & ProjectCode & CntrType & QCType & CollMethod & SampleDepth \\ 
\midrule\addlinespace[2.5pt]
\multicolumn{6}{l}{PIN} \\ 
\midrule\addlinespace[2.5pt]
EBL1-9244 & Baseline & Plastic Bottle & Regular & \cellcolor[HTML]{90EE90}{CO-E} & \cellcolor[HTML]{90EE90}{4} \\ 
EBL1-9238 & Baseline & Plastic Bottle & Regular & GS & 0 \\ 
EBL1-9241 & Baseline & Plastic Bottle & Regular & GS & 0 \\ 
\midrule\addlinespace[2.5pt]
\multicolumn{6}{l}{HETL} \\ 
\midrule\addlinespace[2.5pt]
EBL1-9251 & Baseline & Plastic Bottle & Regular & \cellcolor[HTML]{90EE90}{CO-E} & \cellcolor[HTML]{90EE90}{4} \\ 
EBL1-9247 & Baseline & Plastic Bottle & Regular & \cellcolor[HTML]{90EE90}{CO-E} & \cellcolor[HTML]{90EE90}{4} \\ 
\midrule\addlinespace[2.5pt]
\multicolumn{6}{l}{Msmt} \\ 
\midrule\addlinespace[2.5pt]
EBL1- & Baseline & Msmt & Regular & Msmt & NA \\ 
\bottomrule
\end{longtable}

\hypertarget{result-table}{%
\subsection{Result Table}\label{result-table}}

The result table should allow an easy comparison on the regular and
duplicate samples (QCTypes) for both water temperature and dissolved
oxygen (Const) as a profile. Both pH and secchi cannot be represented as
profiles, but should still be included in this result section. This
specific example does not include duplicates, which is something that
should be flagged. The secchi and pH measurements should also include
duplicates.

\begin{verbatim}
Warning: Values from `Result` are not uniquely identified; output will contain
list-cols.
* Use `values_fn = list` to suppress this warning.
* Use `values_fn = {summary_fun}` to summarise duplicates.
* Use the following dplyr code to identify duplicates.
  {data} %>%
  dplyr::group_by(ProfileDepth, Const, QCType) %>%
  dplyr::summarise(n = dplyr::n(), .groups = "drop") %>%
  dplyr::filter(n > 1L)
\end{verbatim}

\begin{table}

\caption{\label{tbl-results}Results
Table}\begin{minipage}[t]{\linewidth}
\subcaption{\label{tbl-results-1}}

{\centering 

\begin{longtable*}{cccc}
\toprule
ProfileDepth & NA\_Regular & water temperature\_Regular & Dissolved Oxygen\_Regular \\ 
\midrule\addlinespace[2.5pt]
0 &  & 17 & 9.13 \\ 
1 &  & 17 & 9.11 \\ 
2 &  & 17 & 9.1 \\ 
3 &  & 17 & 9.08 \\ 
4 &  & 16.9 & 9.06 \\ 
5 &  & 16.7 & 9.2 \\ 
\bottomrule
\end{longtable*}

}

\end{minipage}%
\newline
\begin{minipage}[t]{\linewidth}
\subcaption{\label{tbl-results-2}}

{\centering 

\begin{longtable*}{rr}
\toprule
pH\_Regular & Secchi\_Regular \\ 
\midrule\addlinespace[2.5pt]
6.78 & 5 \\ 
\bottomrule
\end{longtable*}

}

\end{minipage}%

\end{table}

\hypertarget{filter-table}{%
\subsection{Filter Table}\label{filter-table}}

\hypertarget{tbl-filter}{}
\begin{longtable}{c}
\caption{\label{tbl-filter}Filter Table }\tabularnewline

\toprule
FilterName \\ 
\midrule\addlinespace[2.5pt]
NA \\ 
EBL1-9247-83 \\ 
EBL1-9247-79 \\ 
\bottomrule
\end{longtable}

\hypertarget{controls-summary}{%
\subsection{Controls Summary}\label{controls-summary}}

A section at the beginning or at the end of each file summary report for
a run sampled on a specific day could also summarize all the flags
detected.



\end{document}
